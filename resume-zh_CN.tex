\documentclass[a4paper,8pt,oneside]{resume}
\setmainfont{Ubuntu}
\usepackage{xeCJK}
\usepackage{linespacing_fix} % disable extra space before next section
\usepackage{xcolor}
\usepackage[colorlinks,linkcolor=blue,anchorcolor=blue,citecolor=green]{hyperref}
\usepackage{listings}
 \lstset{
	numbers=left,
	numberstyle=\tiny,
	basicstyle=\footnotesize,
	backgroundcolor= \color{gray},
	keywordstyle=\color{blue!70},
	breaklines=true,
	breakautoindent=true,
	breakindent=4em,
	commentstyle=\color{red!50!green!50!blue50}
	frame=shadowbox,
	framextopmargin=1pt,framexbottommargin=1pt,abovecaptionskip=-1pt,belowcaptionskip=1pt,
	xleftmargin=1em,xrightmargin=1em,
	language=C	
}
  
\begin{document}
\pagenumbering{gobble} % suppress displaying page number

\name{高浩博}
% {E-mail}{mobilephone}{homepage}
\contactInfo{goal.haobo@gmail.com}{(+86)17729734085 }{http://www.vample.info}
 
\section{\faGraduationCap\  教育背景}
\datedsubsection{\textbf{河南科技大学},信息工程 }{2014 -- 2018}
\textit{全日制本科}
信息工程系,软件工程专业

\section{\faUsers\ 实习/项目经历}
\datedsubsection{\textbf{洛阳市恒生杯全国大学生机器人大赛}}{2015年11月 -- 2016年3月}
\role{pwm舵机,syn6288 语音模块 }{团队项目}

\datedsubsection{\textbf{毕业设计}}{2017年12月 -- 2018年6月}
\role{zigbee, TICC2530}{个人项目}


\datedsubsection{\textbf{汉威光电股份有限公司} ,郑州}{2017年9月 -- 2018年6月}
\role{嵌入式软件工程师}{STM32F207,STM32F407}

主要的成果如下:
\begin{itemize}
  \item 优化了内存使用,缓解了内存紧张,为新平台的引入缓解了时间。
  \item 根据高内聚,低耦合的思想, 优化了code,写了一些功能性的接口,添加详细注释,使软件有了框架。引入了大量的数据结构,面向对象的思想,让代码更加容易理解。
  \item 完成了通过文本配置文件(人类可读),启动读取屏参,设置运行参数的设计和实现。
  \item lwip  TCP  socket 通讯。
  \item 上位机程序 ,基于C\# 开发了串口上位机。控制调试串口。
  \item 插播播放表功能的设计和实现
  使用kernel 参照 list.h 以及面向对象的思想,把一个播放任务使用一个结构体来描述,当有插播时,会形成链表。 这样实现切换内容。
  \item 雾区诱导系统的研发
\end{itemize}

\datedsubsection{\textbf{富士康郑州研发中心(ZZDC)} ,郑州}{2018年7月 -- 2019年3月}
\role{BSP驱动工程师}{平台:msm8937,sdm660,sdm710}
\begin{itemize}
  \item 参与boot 部分的debug,和porting. 关键词,uefi,lk,boot\_image, edk2,bootable.实际参与过 Performance debug, 和 android O 升 P boot 部分 porting
  \item 研究 各种平台 的boot 流程。
  \item 研究编译脚本。shell python
  \item 评估和研发单片机项目。
  在室内定位项目上,为了评估uwb 模组的定位准确性,写了一个linux 平台下的 串口数据向服务器上报的例程,这个例程实现了linux 从串口接收数据,然后
  使用tcp 上传到服务器。请参考: \href{https://github.com/haobogao/mid_report}{ 点我 }  
  
\end{itemize}

\datedsubsection{\textbf{个人项目}}{2015年5月 -- 至今}
\role{uboot,linux driver}{个人项目}
\begin{onehalfspacing}
个人利用空余时间 在2440 平台。 对自己比较感兴趣的uboot 和 kernel 进行探索,主要是为了学习,所以产出较少:
\begin{itemize}
\item 当时写的一些2.6 kernel 的一些驱动: \href{https://github.com/haobogao/2440_drv}{ 点我 } 

\item 当时写的一些 裸机code: \href{https://github.com/haobogao/2440_boot}{ 点我 } 

\item 这个是18年12月 购买orangepi 后决定参照 lichee 编译系统 来改进 2440的编译,顺便练习shell,python .  https://github.com/haobogao/2440
\end{itemize}
\end{onehalfspacing}


\section{\faCogs\  技能}
% increase linespacing [parsep=0.5ex]
\begin{itemize}[parsep=0.5ex]
  \item 编程语言: 熟练C语言,熟悉shell,makefile,C\#,C++,latex,了解 python,java,...
  \item 开发: linux设备驱动,单片机开发,linux 应用开发,裸机开发,uboot
  \begin{lstlisting}
#define  熟练   使用熟练,开发中不需要时常查资料
#define  熟悉   进行过相应的开发,开发时需要常查资料
#define  了解	暂时不能进行开发,能读懂相应的code
\end{lstlisting}
\end{itemize}

\section{\faInfo\ 其他}
% increase linespacing [parsep=0.5ex]
\begin{itemize}[parsep=0.5ex]
  \item GitHub: https://github.com/haobogao
  \item 语言: 英语 - 熟练(CET-4)
\end{itemize}

\section{求职意向及简单介绍}

我在15年接触嵌入式,大学专业 软件工程。 在校期间曾经在学长的带领下完成智慧家居机器人的一些工作,大三,16年了解到linux,并投入精力自学之,曾看过LDD
,LKD,understanding the linux kernel .这些书籍,对驱动开发的一些概念比较了解,有一年半以上的企业内嵌入式开发经验,个人比较喜欢linux kernel 相关的一些开发,并定位自己在不远的将来要精通linux 。
我希望在一家重视技术的公司,拿着合乎我能力的薪资,快乐的钻研嵌入式linux技术。   


\end{document}
